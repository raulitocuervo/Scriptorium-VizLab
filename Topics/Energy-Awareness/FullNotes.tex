\begin{abstract}
The exponential growth of connectivity and computing demand has made \gls{nfvi} major contributors to global energy consumption. Conventional network and service deployments, whether based on legacy hardware appliances or \gls{nfvi} stacks, struggle to dynamically provision resources for peak data traffic demand. This results in nearly constant energy consumption even during low-traffic periods, leading to inefficient resource use.
Network softwarization and virtualization have enabled flexible and programmable service deployments, which are beneficial for rapid and dynamic scaling of network functions. This paper validates energy-aware service and network orchestration with a \gls{zsm} framework for autonomous optimization of computing, network, and power resources from \gls{nfvi} on a use case focused on connected mobility, and in particular, smart traffic management. By modeling road traffic based on vehicle count and type and \gls{3gpp} profiles for data formats in vehicular communication scenarios, the \gls{zsm} framework adjusts services and resources to services requirements and to actual demand. Experimental validation on the real-life Smart Highway testbed in Antwerp (Belgium) demonstrates a strong correlation between vehicular traffic and power consumption, supporting the hypothesis that adaptive compute and network resource management reduces unnecessary energy use and advances the vision of sustainable and self optimizing \gls{6g} networks.
\end{abstract}
\glsresetall
\section{Energy-Awareness Experiment 1}\label{awar:1}
This is my starting point \cite{3gpp-ts-22-186,3gpp_ts_26_511_v18_1_0}

Every physical capability of the \gls{rsu} is different from each other in terms of maximum wattage.
When the \glspl{rsu} were stressed based on the amount of vehicles, they have a limit of watts also linked to the CPU consumption.

\begin{table}[htbp]
    \centering
    \caption{Average vehicular counts and power consumption per RSU.}
    \label{tab:rsu-traffic-power}
    \begin{tabular}{|c|cc|cc|}
        \hline
        \textbf{RSU} & \multicolumn{2}{c|}{\textbf{Night (17:00--06:00)}} & \multicolumn{2}{c|}{\textbf{Day (06:00--17:00)}}                        \\
        \cline{2-5}
                     & Vehicles                                           & Power (W)                                        & Vehicles & Power (W) \\
        \hline
        1            & 0                                                  & 59                                               & 4        & 58        \\
        2            & 1                                                  & 55                                               & 7        & 55        \\
        3            & 5                                                  & 56                                               & 23       & 56        \\
        5            & 2                                                  & 51                                               & 7        & 51        \\
        6            & 1                                                  & 60                                               & 5        & 60        \\
        7            & 6                                                  & 61                                               & 16       & 61        \\
        \hline
    \end{tabular}
\end{table}
\begin{table}[!t]
    \centering
    \caption{Location mapping between inductive sensors and \glspl{rsu} to monitor vehicular traffic per \gls{rsu} in the E13 Highway, Antwerp Belgium.}
    \label{tab:location-mapping-rsu}
    \begin{adjustbox}{max width=\columnwidth}
        \begin{tabular}{|l|c|l|l|}
            \hline
            \textbf{\gls{rsu}} & \textbf{Vehicles} & \textbf{RSU Location} & \textbf{Nearest Inductive Sensor} \\
            \hline
            1 & 12 & 51.210575, 4.46655        & 51.21056678, 4.466446158 \\
            3 & 25 & 51.215186667, 4.450568333 & 51.21587307, 4.452993707 \\
            7 & 12 & 51.210616667, 4.480896667 & 51.21066931, 4.480559376 \\
            8 & 26 & 51.211091667, 4.491425    & 51.21127494, 4.491424033 \\
            4 & 16 & 51.21574, 4.457151667     & 51.21577734, 4.457154961 \\
            6 & 16 & 51.211236667, 4.472586667 & 51.21121407, 4.472483657 \\
            \hline
        \end{tabular}
    \end{adjustbox}
\end{table}

Table~\ref{tab:rsu-max-capacity} summarizes the maximum number of users (U) and the corresponding maximum wattage (W) capacity for each \gls{rsu}. As shown, \gls{rsu} 5 supports the highest number of vehicles and reaches a higher power consumption compared to the others. This variation highlights the heterogeneous nature of the \glspl{rsu} in the testbed. Figure~\ref{fig:exp1-figure} visually presents the relationship between the number of vehicles and the power consumption for each \gls{rsu}, illustrating how the power usage increases with the load and reaches a plateau at the maximum capacity.

Max vehicles registered: 39 \gls{rsu} 5
\begin{table}[ht]
    \centering
    \caption{Maximum U and W capacity for each \gls{rsu}.}
    \begin{tabular}{ccc}
        \hline
        \gls{rsu} & Max U & Max W Capacity \\
        \hline
        1         & 27    & 69             \\
        2         & 27    & 67             \\
        3         & 27    & 70             \\
        5         & 36    & 68             \\
        6         & 27    & 77             \\
        7         & 31    & 75             \\
        \hline
    \end{tabular}
    \label{tab:rsu-max-capacity}
\end{table}

\begin{figure}[!htb]
    \centering
    \includegraphics[width=1\columnwidth]{Topics/Energy-Awareness/Figures/power_vehicles_rsu_maximums.pdf}
    \caption{Relationship between the number of vehicles and power consumption for each \gls{rsu}, showing how power usage increases with load and plateaus at maximum capacity.}
    \label{fig:exp1-figure}
\end{figure}

\subsection{Data Analysis and Impact}
A detailed analysis of the collected data, exported to a CSV file, provides deeper insights into the energy consumption patterns of the \glspl{rsu}. The analysis involved calculating the energy efficiency, defined as vehicles per watt, and modeling the power consumption trend using linear regression for each \gls{rsu}. The energy efficiency was computed by dividing the number of vehicles by the corresponding power consumption (in watts) for each data point. The trend was determined by applying a first-degree polynomial fit to the vehicle and power data for each \gls{rsu}, which yields a linear model representing the expected power usage as a function of the number of vehicles.

The data confirms a strong positive correlation between the number of connected vehicles and power consumption across all \glspl{rsu}. The linear trend reveals that as more vehicles connect, the power draw increases, which is expected. However, the rate of this increase varies, highlighting the differences in hardware and configuration among the units. For instance, \gls{rsu}6 shows a steeper power consumption curve compared to \gls{rsu}5, indicating it consumes more power for each additional vehicle.

The impact of these findings is significant for network management:
\begin{itemize}
    \item \textbf{Energy-Aware Load Balancing:} By understanding the unique power profile and efficiency of each \gls{rsu}, a central orchestrator can make smarter, energy-aware decisions. For example, during periods of low network traffic, vehicles can be consolidated onto the most energy-efficient \glspl{rsu}, allowing others to be put into a low-power state.
    \item \textbf{Predictive Resource Management:} The trend models allow for the prediction of power consumption based on anticipated traffic loads. This enables proactive resource allocation and helps prevent over-provisioning, thereby reducing operational costs and the carbon footprint of the infrastructure.
    \item \textbf{Heterogeneity as an Advantage:} The inherent heterogeneity of the \glspl{rsu} can be leveraged as an advantage. High-demand, performance-critical applications can be assigned to the most powerful \glspl{rsu}, while less demanding tasks can be handled by more energy-efficient, lower-capacity units, optimizing the overall performance and energy usage of the system.
\end{itemize}

To illustrate the calculations, consider a data point for \gls{rsu}1 where it serves 6 vehicles and consumes 62.0 watts. The energy efficiency is calculated as:
\[ \text{Vehicles per Watt} = \frac{6 \text{ vehicles}}{62.0 \text{ W}} \approx 0.097 \]
This metric indicates how many vehicles are being served for each watt of power consumed. For the trend analysis, a linear model of the form $P(v) = m \cdot v + c$ is fitted to the data for each \gls{rsu}, where $P$ is the power, $v$ is the number of vehicles, $m$ is the slope (power increase per vehicle), and $c$ is the y-intercept (base power consumption). For \gls{rsu}1, the model might yield a trend value of approximately 60.53 W at 6 vehicles, representing the expected power consumption based on the overall behavior of that specific unit. This allows for a standardized comparison of expected versus actual power draw.
