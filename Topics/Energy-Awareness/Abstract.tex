\begin{abstract}
The exponential growth of connectivity and computing demand has made \gls{nfvi} major contributors to global energy consumption. Conventional network and service deployments, whether based on legacy hardware appliances or \gls{nfvi} stacks, struggle to dynamically provision resources for peak data traffic demand. This results in nearly constant energy consumption even during low-traffic periods, leading to inefficient resource use.
Network softwarization and virtualization have enabled flexible and programmable service deployments, which are beneficial for rapid and dynamic scaling of network functions. This paper validates energy-aware service and network orchestration with a \gls{zsm} framework for autonomous optimization of computing, network, and power resources from \gls{nfvi} on a use case focused on connected mobility, and in particular, smart traffic management. By modeling road traffic based on vehicle count and type and \gls{3gpp} profiles for data formats in vehicular communication scenarios, the \gls{zsm} framework adjusts services and resources to services requirements and to actual demand. Experimental validation on the real-life Smart Highway testbed in Antwerp (Belgium) demonstrates a strong correlation between vehicular traffic and power consumption, supporting the hypothesis that adaptive compute and network resource management reduces unnecessary energy use and advances the vision of sustainable and self optimizing \gls{6g} networks.
\end{abstract}
\glsresetall